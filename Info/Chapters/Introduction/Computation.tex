\subsection{Computer Simulation of Molecular Systems}\label{subsec:computation}

The role of computation in biology, biological chemistry, and biophysics has shown a steady increase over the past few decades. The continuing growth of computing power (in particular in the context of personal computers) has made it possible to analyze, compare, and characterize large and complex data sets that are obtained from experiments on bio-molecular systems. \cite{van2006biomolecular}. Chemical systems are generally too complex to be treated by analytical theoretical methods so it becomes a necessity to implement numerical analysis to study and predict some properties and configurations.

Computer simulations on models of physical systems have been carried out for more than 65 years now \cite{oostenbrink2007applications}. Many developments have turned molecular dynamics simulations into a valuable tool, complementary to experimental investigation, to probe into structure, dynamics, and activity of large biologically relevant molecules. The thermodynamic information that can be obtained from computer simulations allows analysis and understanding of molecular processes and prediction of molecular properties, highly valued information in medical chemistry and drug design. 

