\subsection{Molecular Dynamics Simulation}\label{subsec:MDS}
The equilibration period already produced a short simulation at constant temperature and volume. At this point we want to elongate the simulation to any time window defined by the user, under constant temperature and pressure.

All MD simulations run for this work had a total length of 100 $[ns]$ with a timestep of 2 $[fs]$ using the leapfrog algorithm described in Section \ref{subsubsec:leapfrog}. The SHAKE algorithm \cite{ryckaert1977numerical} was employed to constrain all bonds to their reference values with a relative tolerance of $10^5 [kJ/mol-nm^{2}]$. In order to study only the effects of generating the dry cavity, the solute must be \textbf{positionally restrain} using the same method, to minimizes the changes in entropy related to configurations that can generates the movement of the peptide across the water box. This is accomplished by generated two extra configuration files, in which the positionally restrained (\textbf{*.pos}) atoms and the reference coordinates(\textbf{*.rpr}) are specified, and are generated from the origin coordinate file (\textbf{*.cnf}).

Periodic boundary conditions with a cubic box were applied according to the information provided in Table \ref{table:box_size}.
Non-bonded interactions were computed using a triple range cut-off. Interactions within a short-range cut-off of $0.8 nm$ were computed every time-step, from a pair-list that was generated every 5 steps. At these time points, interactions between $0.8$ and $1.4 nm$ were also computed which were kept constant between these updates. A reaction-field contribution was added to electrostatic interactions approximating for a homogeneous medium outside the $1.4 nm$ cut-off, employing the relative permittivity of SPC water ($\epsilon_w=61$) \cite{tironi1995generalized}. Finally, all simulations were simulated using the isobaric isothermic ensemble (NPT) using the weak-coupling (Berendsen thermostat and barostat) scheme for temperature ($298.15 [K]$) and pressure ($1 [atm]$) control \cite{berendsen1984molecular}. 

