\renewcommand{\baselinestretch}{2}
\setstretch{1.2}

\section*{Abstract}
%\vspace{2cm}
The free energy of solvation $\Delta G^{solv}$ is the energy associated with the process of reorganizing solvent molecules, like water, around a solvated molecule, i.e, the proccess of moving the solute from vaccum  to it's position into the solvent. This process will be thermodynamically favorable only if the overall Gibbs energy of the solution is decreased, compared to the Gibbs energy of the separated solvent and solute. It's also function as an indicator of how probable will be a molecular system, in terms of its configuration, compare to others. 
%This is extremely useful in drug design industries because provides a tool to forecast, throw computer simulations, different versions of synthetic molecules before the manufacture process. 

$\Delta G^{solv}$ can be separated in two components: $\Delta G_{polar}$ and $\Delta G_{non-polar}$. The first is define as the  free  energy  of  creating  the  solute’s  charge  distribution  inside  a pre-existing solute cavity, and the second term, correspond to work required to place an uncharged solute insidethe solvent and generates a dry cavity on it. 

In this work, Molecular Dynamics technique (explicit model) was perform using GROMOS simulation software to calculate the van der Waals $\Delta G_{vdw}$ cavity-formation free energy for 20 alanine peptides divided in extended and helix configurations, ranging from 1 to 10, and compare the results with two impliri solvent models models: the standard model based on the molecular surface of the solute, and a new methodology which has a closer physical meaning modeling the solute as an uncharged capacitor.

An extended overview of the MD method is provided with the methodology for implementing GROMOS simulation software not just to replicate these results, but also for the calculation of the relative free energies of any structure or compound of interest. 
\section*{Resumen}
La energía libre de solvatación $\Delta G^{solv}$ es la energía asociada al proceso de reorganización de las moléculas de disolvente y soluto, en moléculas solvatadas. Este proceso es termodinámicamente favorable solo si la energía libre de Gibbs de la solución es menor, comparada con la energía libre de Gibbs de solvente y molécula por separado. Este parámetro también funciona como un indicador de qué tan probable podría ser una configuración de un sistema molecular, en comparación a otros. 

$\Delta G^{solv}$ puede separarse en dos componentes: $\Delta G_{polar}$ y $\Delta G_{no-polar}$. El primer término se define como la energía libre necesaria para crear la distribución de cargas del soluto en una cavidad pre-existente, y el segundo término corresponde al trabajo requerido para ubicar una molécula de soluto no-cargado en el solvente generando una cavidad seca en este último. 

En este trabajo, se usó la técnica de Dinámica Molecular junto al software de simulación GROMOS para calcular la energía libre de vand der Waals $\Delta G_{vdw}$ necesaria para generar la cavidad para 20 péptidos de alanina divididos en dos tipos de estructuras: hélice y extendida, cuya composición va desde 1 a 10 monómeros. Los resultados fueron contrastados con dos modelos: modelos implíticos tradicionales, y una nueva metodología que tiene una fundación más profunda y un significado físico más cercano a la realidad.
Se presneta también una descripción general de la matodología MD junto con el flujo de información usado por el software GROMOS, no solo para replicar los resutlados de este trabajo, sino que también apra el cálculo de energías libres de cualquier estrucutra molecular de interés. 







